\chapter{素数}
\begin{introduction}[本章内容提要]
	\item 质因数分解
	\item 筛选素数
	\item 大区间问题
	\item 梅森素数
	\item 完全数
\end{introduction}

\section{质因数分解}
\begin{definition}{素数}{prime}
	素数$p$是大于1且它的因数只有1和$p$的整数。
\end{definition}

\begin{property}
	令p是素数,假设p整除乘积ab,则p整除a或p整除b(或者p既整除a也整除b)  
\end{property}

\begin{theorem}{素数整除性质}{label}
假设素数p整除乘积$a_1a_2...a_r$ ,则p整除$a_1 \quad ,a_2 \quad , ...,\quad a_r$ 中至少一个因数。
\end{theorem}

\begin{proof}
	使用上面的性质可以方便的证明该定理。
\end{proof}

\vbox{}

下面将用{\heiti 素数整除性质}证明每个正整数可唯一分解成素数的乘积(算术基本定理)。

\begin{theorem}{算术基本定理}{suanshu}
	每个整数n>=2可唯一分解成素数乘积$n=p_1p_2...p_r$。
\end{theorem}

\begin{proof}
算术基本定理包含两个断言:

断言1:数n可以以某种方式分解成素数的乘积;

断言2:仅有一种这样的因数分解方式(因数重排除外);

- 对于断言1的证明用到归纳证明法 (当N+1是合数时,其必然可以分解成$N+1=n_1*n_2$,而前面已经归纳完毕,即$n1,\ n2$一定可以分解成素数,因此得证)。

- 对于断言2,可以由定理\ref{thm:suanshu}证得。
\end{proof}

\vbox{}

下面考虑如何进行质因数分解。

简单直接的方法是试除2,3,4,5,...分解n,但效率较低。
考虑一个整数n,如果n本身不是素数,则必有整除n的素数$p\leqslant \sqrt{n}$ 。于是我们可以遍历$2\sim \sqrt{n}$
进行试除,时间复杂度为$O(\sqrt{n})$。如果提前预处理出所有素数,则可每次只试除素数,时间复杂度为$O(\frac{\sqrt{n}}{ln \sqrt{n}})$

\lstinputlisting[language=C++, style=codestyle2]{code02/factor.cpp}

\section{区间素数筛}
\subsection{埃氏筛法}

\subsection{欧拉线性筛法}

\subsection{区间数的最大质因子}



\section{区间质因数分解}

\section{大区间素数筛与质因数分解}

\section{梅森素数与完全数}


\vbox{}

\begin{problemset}
	\item 撒旦发射点
	\item 
\end{problemset}