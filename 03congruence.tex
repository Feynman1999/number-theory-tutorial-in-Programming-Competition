\chapter{同余}

\begin{introduction}
\item 同余方程
\item 快速乘、快速幂
\item 逆元、线性求区间逆元
\item 中国剩余定理
\item 欧拉降幂
\item 卡米歇尔数
\item Miller\_Rabin
\item Pollard\_Rho
\item 离散对数
\item 原根
\end{introduction}


\section{同余式与同余方程}
\subsection{同余式}
整除性是很好的性质,这在最大公因数、模线性方程和素数分解中均得到了体现。而同余式提供了一种
描述整除性质的简便方式。

\begin{definition}{同余}{label}
	如果$m|(a-b)$,我们就说a与b模m同余并记之为$a\equiv b(mod\ m)$。
	
	特别地,$a\equiv a\%m(mod\ m)$ 。
\end{definition}

数m叫做同余式的模。 具有相同模的同余式在许多方面表现得很像通常的等式。例如:

若$a_1\equiv b_1(mod\ m) \ ,\quad a_2\equiv b_2(mod\ m)$     则$a_1\pm a_2\equiv b_1\pm b_2(mod \ m)$ ,$a_1a_2\equiv b_1b_2(mod \ m)$ 

{\heiti 但 $ac\equiv bc(mod \ m)$ 时,未必有$a\equiv b(mod \ m)$,只有$gcd(c,m)=1$,才可以消去c。}


\subsection{同余方程}
如果同余式含有未知数,我们考虑如何求解。

首先考虑“穷举法” ,要解模m同余式,可让每个变量试取0,1,2...,m-1。例如,解同余式$x^2+2x-1\equiv 0(mod\ 7)$,
就去试$x=0\quad, x=1\quad,...\quad,x=6$,这样可以求出两个解$x\equiv 2(mod\ 7)$和$x\equiv 3(mod\ 7)$ 当然还有其他解,但新的解与2或3是同余的,如9和10。当我们说“求同余式的所有解时”,是指求所有不同余的解,即相互不同余的所有解。

许多同余式是没有解的,如$x^2\equiv 3(mod\ 10)$。

\vbox{}

下面考虑如何求解同余式$ax\equiv c(mod\ m)$。 
\begin{example}
解同余式$18x\equiv8(mod\ 22)$
\end{example}

等价于求$22\ |\ (18x-8)$,即求$18x-22y=8$。       

解这类方程的问题在第一章中研究过。

对于$ax\equiv c(mod\ m)$ ,其有解{\heiti 当且仅当}线性方程$ax-my=c$ 有解。


由前可知,线性方程$ax-my=c$ 有解的充分必要条件是$gcd(a,m)\ |\ c$。

且方程$au+mv=g$ 一定有解 ,设一个解为$(u_0,v_0)$,则有
$$
a\frac{cu_0}{g}+m\frac{cv_0}{g}=c
$$
这说明$x_0= \frac{cu_0}{g}$是同余式$ax\equiv c(mod\ m)$ 的一个解,通解为$x=x_0+k\cdot \frac{m}{g}$。

{\heiti 由于相差m的倍数的任何两个解认为是相同的},所有恰好有$g$个不同的解,这些解通过取$k=0,1,2...,g-1$而得到。将上述过程概述为定理:


\begin{theorem}{线性同余式定理$ax\equiv c(mod\ m)$}{label}
设a,c与m是整数,m>=1,且设$g=gcd(a,m)$。 

(a)  如果$g\nmid c$,则同余式$ax\equiv c(mod\ m)$ 没有解

(b)  如果$g\ |\ c$ ,则同余式$ax\equiv c(mod\ m)$ 恰好有g个不同的解。要求这些解,首先求线性方程$au+mv=g$ 的一个解$(u_0,v_0)$ 
{\heiti (欧几里得回带法,在计算机上递归求得,称为扩展欧几里得算法)} 。则$x_0=(\frac{cu_0}{g}\% \frac{m}{g} + \frac{m}{g}) \% \frac{m}{g}$是$ax\equiv c(mod\ m)$ 的解,不同余解的完全集由
$$
x\equiv x_0+k\cdot \frac{m}{g} \  (mod\ m),\quad k=0,1,2,...,g-1
$$
给出。
\end{theorem}


例如,同余式$943x\equiv 381(mod\ 2576)$ 无解,这是因为$gcd(943,2576) \nmid 381$。

另一方面,同余式$893x\equiv 266(mod\ 2432) $
有19个解,因为$gcd(893,2432)=19,\  19|266$     这个19即为不同余解的个数。

下面解方程$893u-2432v=19$ ,使用欧几里得回带法可以求得解$(u,v)=(79,29)$ ,乘以266/19=14得方程$893x-2432y=266$的解$(x,y)=(1106,406)$ ,
即1106是同余式方程的一个解,这样的互不同余的解共有19个。1106加上2432/19=128的倍数(\%2432)就可得到完全解集。


\begin{remark}
	线性同余式定理最重要的情形是$gcd(a,m)=1$ ,在这种情形下,同余式恰好有一个解。
\end{remark}

\lstinputlisting[language=C++, style=codestyle2]{code03/modequation.cpp}

\vbox{}

对于非线性的同余式,其解“不是很确定”。

我们熟悉的是对于{\heiti 一个d次实系数多项式的实根不超过d个} ,这个结论对于同余式并不成立。

例如$x^2+x\equiv 0(mod\ 6)$ 有4个模6不同的根:0,2,3,5 

但是,{\heiti 当p为素数时,这个结论依然成立:}

\begin{theorem}{模p多项式根定理}{label}
设p为素数,$f(x)=a_0x^d+a_1x^{d-1}+...+a_d$ 是次数为d>=1的整系数多项式,且p不整除$a_0$ ,则同余式$f(x)\equiv 0(mod\ p)$ 最多有d个模p不同余的解。
\end{theorem}

\section{快速乘与快速幂}
快速乘和快速幂作为工具经常在程序设计竞赛中遇见。
\subsection{快速乘}
在$C++$中,变量最多只能表示到$2^{64}-1$这么大,所以如果我们要计算$a*b\%c$,而$a,b$都是接近表示上限的数,这个时候就需要快速乘。
即将$b$按二进制位分解分别加上,时间复杂度为$O(log(b))$。
\lstinputlisting[language=C++, style=codestyle2]{code03/fastmul.cpp}


\subsection{快速幂}
现在我们要计算$a^b\%c$,如果乘$b$次,时间复杂度太高,考虑将$b$按照二进制分解,每一位分别计算并乘在一起即可。
时间复杂度为$O(log(b))$。相比于快速乘,只是加法变了乘法。
\lstinputlisting[language=C++, style=codestyle2]{code03/fastexp.cpp}


\section{费马小定理与逆元}
前面我们讨论了关于同余式、同余方程的一些性质,小结一下,互质这个条件很重要。

\begin{itemize}
	\item 对于$ac\equiv bc(mod \ m)$,如果$gcd(c,m)=1$,则可以消去c,得到$a\equiv b(mod \ m)$ 。
	\item 对于同余方程 $ax\equiv c(mod\ m)$,若$gcd(a,m)=1$ ,同余式恰好有一个解。
\end{itemize}

{\heiti 对于式子$ax\equiv c(mod\ m)$,令$c=1$,得$ax\equiv 1(mod\ m)$,若$gcd(a,m)=1$,则方程有唯一解$x_0$,
我们称$x_0$为$a$在模$m$意义下的逆元,常记作$a^{-1}$。}

逆元可以用来说明一些事情,比如如果$ac\equiv bc(mod\ m)$,若$gcd(c,m)=1$,
则存在$c^{-1}$使得$cc^{-1}\equiv 1 (mod\ m)$。所以可以对$ac\equiv bc(mod\ m)$两边同时乘以$c^{-1}$,得到$a\equiv b(mod\ m)$。

如何求解逆元呢?拓展欧几里得即可,因为就是一个同余方程:
\lstinputlisting[language=C++, style=codestyle2]{code03/inverse.cpp}

\vbox{}

若模数是素数,我们还可以用费马小定理求解。

\begin{theorem}{费马小定理}{femat}
	设p是素数,a是任意整数且$p\nmid a$ ,则 $a^{p-1}\equiv 1(mod\ p)$。 
\end{theorem}

在证明费马小定理之前,先来看一个引理。

\begin{lemma}{为证明费马小定理做准备}{forfemat}
	设$p$是素数,a是任何整数且 $p\nmid a$ 则数
	$$
	a,2a,3a,...,(p-1)a\qquad (modp)
	$$
	与数
	$$
	1,2,3,...,(p-1)\qquad (modp)
	$$
	相同,尽管它们的次序不同。
\end{lemma}

\begin{proof}
	数列$a,2a,3a,...,(p-1)a$,包含$p-1$个数,显然没有一个数被$p$整除 ,假设从中取出两个数ja和ka是关于p同余的,即$p|(j-k)a$,
	又p是素数且p不整除a,所以p整除$(j-k)$。但是$|j-k|<p-1 $,所以$j-k=0$,即$j=k$。
	
	这表明,这p-1个数模p不同,由于任何数mod p仅有$p-1$个不同的非零值,证毕。
\end{proof}

\begin{proof}
	费马小定理的证明。
	
	利用该引理\ref{lem:forfemat},即可完成对费马小定理\ref{thm:femat}的证明,将上面引理列到的两组数相乘,可得到
	$$
	a^{p-1} \cdot (p-1)!\equiv (p-1)!   \qquad (modp)
	$$
	由于$(p-1)!$与p互质(显然除了1没有其它公共因子了),可以消去它(本节开头有提到这个性质),则$a^{p-1}\equiv 1(mod\ p)$。
\end{proof}

\vbox{}

有了费马小定理,若模数$m$是质数,则数$a$关于$m$的逆元就是$a^{m-2}$,因为$a* a^{m-2} \equiv 1 \ (mod p)$。
使用快速幂直接计算即可。

\vbox{}

{\heiti 使用费马小定理还可以进行素数测试},后面小节会提到。

\section{中国剩余定理}

\begin{theorem}{中国剩余定理}{CRT}
	设m与n是整数,$gcd(m,n)=1$,b与c是任意整数,则同余式组
	$x\equiv b(mod\ m)$与$x\equiv c(mod\ n)$恰有一个解$0\leqslant x<mn$。
\end{theorem}

\begin{proof}
	对于第一个同余式$x\equiv b(mod\ m)$,其解由形如$x = my +b$的所有数组成。将其带入第二个方程可得
	$my\equiv c-b (mod\ n)$,已知$gcd(m,n)=1$,由线性同余式定理知其恰有一个解$y_1,\ 0\le y_1<n$,则$x_1=my_1+b$
	给出了原来同余式组的解,这是在$[0,mn)$之间的唯一解。
\end{proof}

\vbox{}

上面只考虑了两个同余式,如果有多个呢?

\begin{custom}{问题}
	求出方程组$x\equiv a_i(mod \ m_i) (0 \leqslant i <n) $ 的解$x$,其中$m_0,m_1,m_2,m_3...m_{n-1}$ 两两互质。
\end{custom}

\begin{solution}
	令$M_i=\prod_{j\neq i}m_j$     则有$(M_i,m_i)=1$   
	
	故存在$p_i,q_i$ ,使得$M_i*p_i+m_i*q_i=1$ 
	
	令  $e_i=M_ip_i$,\quad    $p_i$即为$M_i$ 模$m_i$下的逆元。   
	
	则有
$$
e_i\equiv\left\{\begin{matrix}
0(mod\ m_j),j\neq i\\ 
1(mod\ m_j),j=i
\end{matrix}\right.	
$$	
	故$e_0a_0+e_1a_1+e_2a_2+...+e_{n-1}a_{n-1}$是方程的一个解。
	
	由中国剩余定理知,$[0\sim\prod_{i=0}^{n-1}m_i\}$ 中必有一解,将上式模$\prod_{i=0}^{n-1}m_i$即可。 
\end{solution}

时间复杂度  $O(nlogm)$,$n$个方程。

\lstinputlisting[language=C++, style=codestyle2]{code03/crt.cpp}

\section{欧拉公式与欧拉降幂}


\section{素性测试}


\section{Pollard\_Rho质因数分解}


\section{离散对数}


\section{原根}




\begin{problemset}
	\item xx
\end{problemset}